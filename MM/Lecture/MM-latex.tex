\documentclass{article}
\usepackage{amsmath}
\usepackage{amssymb}
\usepackage{graphicx}
\usepackage{geometry}
\usepackage{multicol}

\title{Microprocessors and Micro controllers \\ CSE3815}
\author{Fairuz Bilquis Khan}
\date{April 2024}

\begin{document}
\maketitle
\newpage
\section*{Evolution of Microprocessors}
There should be overview chart of difference between intel 4004 to pentium 4.
\newpage
\subsection*{Microprocessors:}
\begin{verbatim}
    `-->Is a semiconductor device consisting of electronics logic circuit.
    `--->Manufactured by using various fabrication schemes.
    `---->Capable of performing computing functions.
    `----->Capable of transporting data/information.

    Can be divided into 3 segments:
        (-)Arithmetic adn logic unit
        (-)Registers unit
        (-)Control unit
\end{verbatim}

\subsection*{Von Neumann Machine:}
\begin{verbatim}
    Von Neumann machine may refer to: Von Neumann architecture, a conceptual model
    of nearly all computer architecture

    Three key concepts:
    (-)Data and instructions are stored in a single set of read-write memory.
    (-)Contents of memory are accessed by memory address without regard to the type
        of data.
    (-)Execution occurs in a sequential fashion, unless explicitly stated otherwise.
\end{verbatim}

\subsection*{Computer System Components:}
\begin{verbatim}
    (-) Memory:
            Stores data and instructions.
    (-) Input/Output:
            Called peripherals.
            Used to input and output data and instructions.
    (-) Arithmetic and Logic Unit:
            Performs arithmetic(+, -)  operations.
            Performs logical(AND, OR, XOR, Shift, Rotate) operations.
    (-) Control Unit:
            Coordinates the execution of instructions by managing the flow of data 
                between the CPU's components.
    (-) System Interconnection and Interaction:
            (+)BUS: A group of lines used to transfer bits between µp and other components.
                    Bus is used to communicate between parts of the computer. There is only
                    one transmitter at a time and only the addressed device can respond.
                    ^\types:
                        >> address
                        >> data
                        >> control
   
        \end{verbatim}
\newpage
\subsection*{CPU Components:}
\begin{verbatim}
    (-) Registers:
            Small, fast storage in the CPU for data, instructions and other items.
            Includes the program counter (PC) and memory address registers (MAR).
            These registers need to match the width of the address bus for proper 
                memory access.
            Data registers should match memory word size for efficient data transfer.
    (-) Control Unit:
            Generates control signals which are necessary for execution of instructions.
            Connect registers to the bus.
            Controls the data flow between CPU and peripheral components.
            Provides status, control and timing signals required for the operations of
                memory and I/O devices.
            Acts as a brain of computer system.
            All actions of the control unit are associated with the decoding and execution
                of instructions(fetch and execute cycles).
    (-) Arithmetic and Logic Unit:
            Execute arithmetic and logical operations.
            Accumulator is a special 8-bit register associated with ALU. Register 'A' in
                8085 is an accumulator.
            Source of one of the operands of an arithmetic or logical operation serves as
                one input to ALU.
            Final result of an arithmetic or logical operation is placed in the accumulator.
\end{verbatim}

\subsection*{Arithmetic and Logic Unit:}
\begin{verbatim}
    ALU performs the following arithmetic and logical operations:
        >> Addition
        >> Subtraction
        >> AND
        >> OR
        >> XOR
        >> NOT
        >> Increment
        >> Decrement
        >> Left Shift, Right Shift, Rotate .
        >> Clear etc.
\end{verbatim}
\newpage

\begin{verbatim}
    Status Flag:
        Intel 8085 microprocessor contains five flip-flops to serve as status flags.
        These flip-flops are set or reset according to the conditions which arise due
            to an arithmetic & logical operation.

            Sign Flag (S): Indicates the sign of the result after an arithmetic operation.
            Zero Flag (Z): Indicates whether the result of an operation is zero.
            Auxiliary Carry Flag (AC): Indicates use for decimal arithmetic operations.
            Parity Flag (P): Indicates the parity of the result.
            Carry Flag (CY): Indicates whether there is a carry-out from the most 
                                significant bit.
\end{verbatim}

\textbf{next is difference between 8085 and 8086 architecture diagram (will add later . not needed for Ct)}
\newpage

\end{document}

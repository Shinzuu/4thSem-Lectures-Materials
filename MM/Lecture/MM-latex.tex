\documentclass{article}
\usepackage{amsmath}
\usepackage{amssymb}
\usepackage{graphicx}
\usepackage{geometry}
\usepackage{multicol}

\title{Microprocessors and Micro controllers \\ CSE3815}
\author{Fairuz Bilquis Khan}
\date{April 2024}

\begin{document}
\maketitle
\newpage
\section*{Evolution of Microprocessors}
There should be overview chart of difference between intel 4004 to pentium 4.
\newpage
\subsection*{Microprocessors:}
\begin{verbatim}
    `-->Is a semiconductor device consisting of electronics logic circuit.
    `--->Manufactured by using various fabrication schemes.
    `---->Capable of performing computing functions.
    `----->Capable of transporting data/information.

    Can be divided into 3 segments:
        (-)Arithmetic adn logic unit
        (-)Registers unit
        (-)Control unit
\end{verbatim}

\subsection*{Von Neumann Machine:}
\begin{verbatim}
    Von Neumann machine may refer to: Von Neumann architecture, a conceptual model
    of nearly all computer architecture

    Three key concepts:
    (-)Data and instructions are stored in a single set of read-write memory.
    (-)Contents of memory are accessed by memory address without regard to the type
        of data.
    (-)Execution occurs in a sequential fashion, unless explicitly stated otherwise.
\end{verbatim}

\subsection*{Computer System Components:}
\begin{verbatim}
    (-) Memory:
            Stores data and instructions.
    (-) Input/Output:
            Called peripherals.
            Used to input and output data and instructions.
    (-) Arithmetic and Logic Unit:
            Performs arithmetic(+, -)  operations.
            Performs logical(AND, OR, XOR, Shift, Rotate) operations.
    (-) Control Unit:
            Coordinates the execution of instructions by managing the flow of data 
                between the CPU's components.
    (-) System Interconnection and Interaction:
            (+)BUS: A group of lines used to transfer bits between µp and other components.
                    Bus is used to communicate between parts of the computer. There is only
                    one transmitter at a time and only the addressed device can respond.
                    ^\types:
                        >>address
                        >>data
                        >>control
   
        \end{verbatim}
\newpage
\subsection*{CPU Components:}
\begin{verbatim}
    (-) Registers:
            Small, fast storage in the CPU for data, instructions and other items.
            Includes the program counter (PC) and memory address registers (MAR).
            These registers need to match the width of the address bus for proper 
                memory access.
            Data registers should match memory word size for efficient data transfer.
    (-) Control Unit:
            

\end{verbatim}


\end{document}

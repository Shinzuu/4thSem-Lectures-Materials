\documentclass{article}
\usepackage{amsmath}
\usepackage{amssymb}
\usepackage{graphicx}
\usepackage{geometry}
\usepackage{multicol}

\title{Algorithms \\ CSE 2415}
\author{Md. Hasan}
\date{April 2024}

\begin{document}
\maketitle
\newpage

\section*{Introduction}

Algorithm is a sequence of steps / step-by-step procedure to solve a problem.

\vspace*{20pt}

Properties of Algorithm:

\begin{itemize}
    \item Specific input
    \item Specific output
    \item Definiteness
    \item Finiteness
    \item Effectiveness
\end{itemize}

\section*{Time and Space Complexity}

\textbf{Examples: }

\hrulefill
\begin{multicols}{3}
    \begin{minipage}{\linewidth}
        \section*{}
        % Your content for section 1 goes here
        \begin{verbatim}
Algorithm:

int i, j;
for(i = 0; i < n; i++){
    for(j = 0; j < n; j++)
        printf(" \%d ", i+j);
}
        \end{verbatim}
    \end{minipage}
    
    \begin{minipage}{\linewidth}
        \section*{}
        % Your content for section 2 goes here
        \begin{verbatim}
        space complexity
        cost      repeat      total
         1          1           1
       1+1+1    1+(n+1)+n     2n+2
       1+1+1  (1+(n+1)+n)+n  2n^2+2n
         1         n^2         n^2

        F(n)= 3n^2 + 3n + 1
        TC -> O(n^2)
        \end{verbatim}
    \end{minipage}
    
    \begin{minipage}{\linewidth}
        \section*{}
        % Your content for section 3 goes here
        \begin{verbatim}
            Space complexity
            cost  repeat  total
            i= 4     1     4
            j= 4     1     4
            n= 4     1     4
            

            S(n)= 12
            SC -> O(1)
        \end{verbatim}
    \end{minipage}
\end{multicols}

\newpage
\hrulefill
    \begin{verbatim}
    Algorithm:

    int i, j, n, A[i][j], B[i][j], C[i][j];
    for(i = 0; i < n; i++){
        for(j = 0; j < n; j++)
            C[i][j] = A[i][j] + B[i][j];
    }
    
    space complexity                    Space complexity

    cost      repeat       total         cost  repeat  total
      1          1           1           i=4     1       4
    1+1+1    1+(n+1)+n      2n+2         j=4     1       4
    1+1+1   n+n(n+1)+n^2   2n^2+2n       n=4     1       4
      1         n^2         n^2         A[][]   4*n*n   4n^2
                                        B[][]   4*n*n   4n^2
                                        C[][]   4*n*n   4n^2

    F(n)= 3n^2 + 4n + 3                 S(n)= 12n^2 + 12
     TC -> O(n^2)                        SC -> O(n^2)
\end{verbatim}

\hrulefill
\begin{verbatim}
    Algorithm:

    int i, n;
    for(i = 0; i < n; i++)
        printf(" \%d ", 2*i);
    
    space complexity                    Space complexity

    cost        repeat      total         cost  repeat  total
      1           1           1           i=4     1       4
    1+1+1  1+(n/2)+1+(n/2)   n+2          n=4     1       4
      1          n/2         n/2         

    F(n)= (3n/2) + 3                    S(n)= 8
     TC -> O(n)                          SC -> O(1)
\end{verbatim}

\newpage
\hrulefill
\begin{verbatim}
    Algorithm:

    int p=0, i, n;
    for(i = 1; i <= p; i++)
        p+=1;

    Step analysis:
    i       p
    0      0+1
    1     0+1+2
    2    0+1+2+3
    3   0+1+2+3+4
    .       .
    .       .
    .       .
    k 0+1+2+3+4+...+k = k*(k+1)/2

    assume , p>n where step number is k and p = k*(k+1)/2
            
\end{verbatim}



\end{document}
